\documentclass[10pt]{beamer}
\usepackage[utf8]{inputenc}
\usetheme{Warsaw} 
\usecolortheme{crane} 
\usefonttheme[onlymath]{serif} % changes font for math text

\title{A Friendly Reminder: Polynomial Equations}
\subtitle{(One among infinitely many) Solution to the Exercise}
\author{Andrea Di Primio}
\date{\today \:(or whatever day it will be)}
\begin{document}

\begin{frame}[plain]
	\maketitle
\end{frame}

\begin{frame}{What is a polynomial equation?}

\begin{definition}[Polynomial equation]
	A polynomial equation is an equality between two polynomials. 	
\end{definition}
\begin{definition}[Degree of a polynomial equation]
	The degree of a polynomial equation of the form $P(x,y,...) = 0$ is defined as the degree of $P(x,y,...)$.
\end{definition}
\vspace{2\baselineskip}
\begin{minipage}{0.48\linewidth}
	\[ x^3 = x^3 + 4x + 7y - 1 \]
\begin{center}
	Two-variable, $1^{st}$ degree\\ polynomial equation
\end{center}
\end{minipage}
\begin{minipage}{0.48\linewidth}
	\[ x^5 + 7x = 0 \]
\begin{center}
	One-variable, $5^{th}$ degree\\ polynomial equation
\end{center}
\end{minipage}
\end{frame}

\begin{frame}{Quadratic equations}
A polynomial equation of degree 2 is also called quadratic equation.\\[\baselineskip]
Assuming to work with one complex variable $x$, the most general quadratic equation is
\[ ax^2 + bx + c = 0 \]
where $a, b, c$ are given complex numbers such that $a \neq 0$, called coefficients.\\[\baselineskip]
\begin{theorem}[Solutions of a quadratic equation I]
	A quadratic equation in one complex variable has exactly two complex solutions.
\end{theorem}
\end{frame}

\begin{frame}{Solving the quadratic equation}
\begin{theorem}[Solutions of a quadratic equation II]
	Given three complex quantities $a,b,c$, with $a \neq 0$, the quadratic equation
	\[ ax^2 + bx + c = 0 \]
	has the two (complex) solutions
	\[ x_{1,2} = \dfrac{-b \pm \sqrt{b^2 - 4ac}}{2a} \]
	and no other ones.
\end{theorem}
\end{frame}

\begin{frame}{A different approach to root calculation}
\begin{theorem}[Vieta's formulas for quadratic polynomials]
	Given a polynomial equation of degree 2 with complex coefficients
	\[ ax^2 + bx + c = 0 \]
	let $x_1, x_2$ be its solutions. Then the following relations hold:
	\[ \begin{cases}
		x_1 + x_2  = -\dfrac{b}{a}\\
		x_1x_2 = \dfrac{c}{a}\\
	\end{cases}\]
\end{theorem}
\end{frame}

\begin{frame}{An example}
	Let us solve the equation:
	\[ ix^2 + \sqrt{3}x - 1 = 0 \]
	Using the formula to obtain solutions:
	\[ x_{1,2} = \dfrac{-\sqrt{3} \pm \sqrt{3 + 4i}}{2i} \]
 	which can be further simplified (do it!) to find
 	\[ \begin{cases}
 	 x_1 = \dfrac{1}{2} + i\left(\dfrac{\sqrt{3}}{2} - 1 \right)\\
 	 x_2 = -\dfrac{1}{2} + i\left(\dfrac{\sqrt{3}}{2} + 1 \right)\\
 	\end{cases}\]
\end{frame}
\appendix
%Consider inserting long proofs in due detail in the appendix. Presentations including tedious derivations tend to be boring!
\end{document}