\documentclass[10pt, twoside]{book}
\usepackage[utf8]{inputenc}
\usepackage[T1]{fontenc}
\usepackage{amsmath, amssymb, amsthm}
\usepackage{amsfonts}
\usepackage{amssymb}
\usepackage{graphicx, float, caption, subcaption}
\usepackage{multirow, booktabs}
\usepackage[pass]{geometry}
\usepackage[italian]{babel}
\usepackage[sortcites=true, style=numeric, sorting=nty, backend=bibtex]{biblatex}
\usepackage{mypackage}
\addbibresource{sources.bib}
%\usepackage[left=2.00cm, right=2.00cm, top=2.00cm, bottom=2.00cm]{geometry}
\author{Andrea Di Primio}
\title{Lezione 3: Esercizi}
\date{9 dicembre 2021}
\newtheorem{prop}{Proposition}[section]
\newcommand{\myint}[4]{\int_{#1}^{#2} #3 \: \mathrm{d}#4}
\newenvironment{esercizio}[1][Esercizio]{\vspace{\baselineskip} \begin{tabular}{|p{\textwidth}|} \hline\textbf{#1}. }{ \\ \hline\end{tabular} \vspace{\baselineskip}}
\begin{document}
	\begin{titlepage}
		\newgeometry{margin=2cm}
		\thispagestyle{empty}
		\maketitle
	\end{titlepage}

	\frontmatter
	\chapter*{Abstract} 
	Abstract in inglese.
	\chapter*{Sommario} 
	Sommario in italiano.
	\chapter*{Ringraziamenti}
	Ringraziamenti.
	\tableofcontents
	\listoffigures
	\listoftables
	
	\mainmatter
	\chapter{Primo capitolo}
	Un riferimento a \cite{testart21}, \cite{testbook21}, \cite{testbook21, testart21}.
	\section{Prima sezione del primo capitolo}
	\begin{equation*}
		\myint{a}{b}{f(x)}{x}
	\end{equation*}
	\begin{esercizio}
		 Calcolare i valori di interi positivi di $n$ per cui $n^2-14n+32$ è un numero primo.
	\end{esercizio}
	\begin{esercizio}[Esercizio più difficile]
		Calcolare i valori interi positivi di $n$ per cui $n!$ termina con esattamente 1000 zeri.
	\end{esercizio}
	\mycmd{Un pacchetto personalizzato}
	\section{Seconda sezione del primo capitolo}
	\section{Terza sezione del primo capitolo}
	\input{chapters/chapter.tex}
	
	\appendix
	\chapter{Prima appendice}
	\chapter{Seconda appendice}
	
	\backmatter
	%\nocite{*}
	\printbibliography[heading=bibintoc, title=Bibliografia]
	
\end{document}