\documentclass{article}

\usepackage[latin1]{inputenc}
\usepackage[italian]{babel}
\usepackage{amsmath}
\usepackage{amssymb}
\usepackage{amsthm}

\newtheorem{theorem}{Teorema}
\newtheorem{definition}{Definizione}
\newtheorem{lem}{Lemma}

\title{titolo}
\author{autori}

\begin{document}

\maketitle

\begin{definition}
	Questa � una definizione.
\end{definition}

\begin{theorem}
	Questo � un teorema.
\end{theorem}

\begin{proof}
	Questa ne � la dimostrazione.
\end{proof}

\begin{theorem}
    Un secondo teorema.
\end{theorem}

\begin{proof}
	Attenzione quando si finisce con una formula:
	\[
		x+y. \qedhere
	\]
\end{proof}
\noindent Altrimenti sono guai!

\begin{lem}[titolo del lemma]
Questo ambiente \`e diverso dagli altri.
\end{lem}
\begin{proof}
Basta osservare la diversa numerazione.
\end{proof}
\end{document}
