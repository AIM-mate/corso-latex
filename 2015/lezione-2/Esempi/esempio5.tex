\documentclass{article}

\usepackage[latin1]{inputenc}
\usepackage[italian]{babel}
\usepackage{amsmath}
\usepackage{amsthm}

\newtheorem{theorem}{Teorema}
\newtheorem{definition}{Definizione}
\newtheorem{lem}{Lemma}

\begin{document}
\section{Array e matrici}
Iniziamo a lavorare con le cose serie. Ecco un sistema da risolvere:

\[
  \left(
	\begin{array}{c c c}
	  1 & 2 & 3 \\
	  4 & 5 & 6 \\
	  5 & 7 & 9
	\end{array}
   \right) 
   \left[
	 \begin{array}{c}
	 x_1 \\ x_2 \\ x_3
	 \end{array}
   \right] 
   = 
  \left[
  \begin{array}{c}
    b \\ bb \\ bbb
  \end{array}
  \right]
\]
Oppure con le \verb|\matrix|:
\[
  \begin{pmatrix}
    1 & 2 & 3 \\
    4 & 5 & 6 \\
    5 & 7 & 9
  \end{pmatrix}
  \begin{bmatrix}
    x_1 \\ x_2 \\ x_3
  \end{bmatrix}
  =
  \begin{bmatrix}
    b \\ bb \\ bbb
  \end{bmatrix}
\]
con una spaziatura migliore e una sintassi pi� facile.

Queste sono le varie possibilit�:
$$
\begin{matrix} 
a & b \\
c & d 
\end{matrix}
\quad
\begin{pmatrix} 
a & b \\
c & d 
\end{pmatrix}
\quad
\begin{bmatrix} 
a & b \\
c & d 
\end{bmatrix}
\quad
\begin{vmatrix} 
a & b \\
c & d 
\end{vmatrix}
\quad
\begin{Vmatrix} 
a & b \\
c & d 
\end{Vmatrix}
$$

\section{Teoremi, definizioni, \dots}
\begin{definition}
	Questa � una definizione.
\end{definition}

\begin{theorem}
	Questo � un teorema.
\end{theorem}

\begin{proof}
	Questa ne � la dimostrazione.
\end{proof}

\begin{theorem}
    Un secondo teorema.
\end{theorem}

\begin{proof}
	Attenzione quando si finisce con una formula:
	\[
		x+y. \qedhere
	\]
\end{proof}
\noindent Altrimenti sono guai!

\begin{lem}[titolo del lemma]
Questo ambiente \`e diverso dagli altri.
\end{lem}
\begin{proof}
Basta osservare la diversa numerazione.
\end{proof}
\end{document}
