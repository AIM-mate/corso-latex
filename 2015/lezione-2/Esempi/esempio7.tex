\documentclass{article}

\usepackage[latin1]{inputenc}
\usepackage[italian]{babel}
\usepackage{amsmath}
\usepackage{array}
\usepackage{booktabs} % per toprule, midrule, ecc.

\begin{document}
\section*{Tabelle}

\begin{table}[h!]
   \centering
   \begin{tabular}{| c c c c | r |}
      \cline{1-4}
      \multicolumn{4}{|c|}{Valori} & \multicolumn{1}{l}{Somma} \\
      \hline
      7 & 5 & 3 & 4 & 19 \\
      2 & 1 & 3 & 3 &  9 \\
      \hline
   \end{tabular}
   \caption{Somme}
   \label{tab:somme}
\end{table}

\vspace{20pt}

\begin{table}[h!]
   \centering
	\begin{tabular}{| r | l || c c | r |}
	   \hline
	   Nome & Cognome & crediti & media & matricola \\
	   \hline
	   Mario & Rossi & 25 & 28,3 & 572139 \\
	   \hline
	   Paolo & Verdi & 80 & 24 & 554291 \\
	   \hline
	   Giuseppe & Re & 112,5 & 22 & 537295 \\
	   \hline
	\end{tabular}
\end{table}

\vspace{20pt}

\begin{table}[h!]
   \centering
	\begin{tabular}{rl|cc|r} 
	% provate anche con 
	%\begin{tabular}{@{}rlccr@{}}
	    \toprule
	        Nome & Cognome & crediti & media & matricola \\
	    \cmidrule(r){1-2}
	    \cmidrule(rl){3-4}
	    \cmidrule(l){5-5}
	        Mario & Rossi & 25 & 28,3 & 572139 \\
	        Paolo & Verdi & 80 & 24 & 554291 \\
	        Giuseppe & Re & 112,5 & 22 & 537295 \\
	    \bottomrule
	\end{tabular}
\end{table}

\end{document}
