\documentclass[]{article}

\usepackage{amsmath}

\begin{document}
\section*{Allineamento equazioni}
	\begin{align}
		\sin(x) & \leq 1 \\
		\arctan(+\infty) & = \frac{\pi}{2}
	\end{align}

	\begin{subequations}
		\begin{align}
			\sin(x) & \leq 1 \\
			\arctan(+\infty) & = \frac{\pi}{2}
		\end{align}
	\end{subequations}

	\[
		\begin{aligned}
			\sin(x) & \leq 1 \\
			\arctan(+\infty) & = \frac{\pi}{2}
		\end{aligned}
	\]

	\begin{alignat}{3}
		\sin(x) & \leq &  1 \\
		\arctan(-\infty) & = & - \frac{\pi}{2}
	\end{alignat}

	\[
		\begin{alignedat}{3}
			\sin(x) & \leq & 1 \\
			\arctan(-\infty) & = & - \frac{\pi}{2}
		\end{alignedat}
	\]

	\begin{equation}
		f(x) = 
		\begin{cases}
		x^3 & \text{se } x \geq 0 \\
		0 & \text{se } x < 0
		\end{cases}
	\end{equation}

\end{document}
