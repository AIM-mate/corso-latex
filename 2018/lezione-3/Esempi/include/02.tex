\section{Tabelle\dots}

\begin{tabular}{rlccr} % provate anche con \begin{tabular}{@{}rlccr@{}}
    \toprule
        Nome & Cognome & crediti & media & matricola \\
    \cmidrule(r){1-2}
    \cmidrule(rl){3-4}
    \cmidrule(l){5-5}
        Mario & Rossi & 25 & 28,3 & 572139 \\
        Paolo & Verdi & 80 & 24 & 554291 \\
        Giuseppe & Re & 112,5 & 22 & 537295 \\
    \bottomrule
\end{tabular}

\section{\dots e array}

Iniziamo a lavorare con le cose serie. Ecco un bel sistema da risolvere:

\[
  \left(
  \begin{array}{c c c}
    1 & 2 & 3 \\
    4 & 5 & 6 \\
    5 & 7 & 9
  \end{array}
  \right) \left[
  \begin{array}{l}
    x_1 \\ x_2 \\ x_3
  \end{array}
  \right] = \left[
  \begin{array}{r}
    b \\ bb \\ bbb
  \end{array}
  \right]
\]
Oppure con le \verb|\matrix|:
\[
  \begin{pmatrix}
    1 & 2 & 3 \\
    4 & 5 & 6 \\
    5 & 7 & 9
  \end{pmatrix}
  \begin{bmatrix}
    x_1 \\ x_2 \\ x_3
  \end{bmatrix}
  =
  \begin{bmatrix}
    b \\ bb \\ bbb
  \end{bmatrix}
\]
con una spaziatura migliore e una sintassi più facile.
