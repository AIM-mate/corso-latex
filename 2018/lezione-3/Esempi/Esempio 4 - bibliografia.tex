\documentclass{article}

\usepackage[italian]{babel}

\title{titolo}
\author{autori}

\begin{document}

\maketitle

\tableofcontents

\section{Un esempio}

Serve aver studiato \cite[Cap. 2]{quarteroni} per capire di cosa stiamo parlando. \\[\baselineskip]
\textbf{Bibliografia 1:}
\begin{thebibliography}{9} %numero di elementi
\addcontentsline{toc}{section}{Prima Bibliografia}
\bibitem{mox21}
    L. Formaggia, A. Veneziani, \emph{Reduced and multiscale models for the human cardiovascular system},
    Technical report, Politecnico di Milano (2003).
    Collection of two lecture notes given at the VKI Lecture Series 2003-2007.
\bibitem{quarteroni}
    A. Quarteroni, \emph{Modellistica numerica per problemi differenziali},
    Springer. Quarta edizione (2008).
\end{thebibliography}

\textbf{Bibliografia 2:}
\begin{thebibliography}{For.Ven. 2003} %etichetta più lunga
\addcontentsline{toc}{section}{Seconda Bibliografia}
	\bibitem[For.Ven. 2003]{mox21}
    	L. Formaggia, A. Veneziani, \emph{Reduced and multiscale models for the human cardiovascular system},
        Technical report, Politecnico di Milano (2003).
        Collection of two lecture notes given at the VKI Lecture Series 2003-2007.
	\bibitem[Quart. 2008]{quarteroni}
    	A. Quarteroni, \emph{Modellistica numerica per problemi differenziali},
        Springer. Quarta edizione (2008).
\end{thebibliography}

\end{document}
